\documentclass{article} % For LaTeX2e
% SIAM Article Template
%\documentclass[review,onefignum,onetabnum]{siamonline190516}

\usepackage{pgf}
\usepackage{tikz}

\usepackage[utf8]{inputenc} % allow utf-8 input
\usepackage[T1]{fontenc}    % use 8-bit T1 fonts
\usepackage{booktabs}       % professional-quality tables
\usepackage{amsfonts}       % blackboard math symbols
\usepackage{nicefrac}       % compact symbols for 1/2, etc.
\usepackage{microtype}      % microtypography

\usepackage{graphicx}
\usepackage{theorem,ifthen,algorithm,algorithmic}
\usepackage{amssymb,amsmath,latexsym,dsfont}
%\usepackage{tikz,pgflibraryplotmarks}
\usepackage{mathrsfs}
%\usepackage{fullpage}
\usepackage{hyperref}
\usepackage{url}
\usepackage{fullpage}


% Optional math commands from https://github.com/goodfeli/dlbook_notation.
%\input{math_commands.tex}


\title{Take home midterm - 213  }

\author{Eldad Haber\\
Department of EOAS\\
The University of British Columbia\\
Vancouver, BC V6T1Z4, Canada \\
\texttt{ehaber@eoas.ubc.ca} 
}


\newcommand{\fix}{\marginpar{FIX}}
\newcommand{\new}{\marginpar{NEW}}

%\iclrfinalcopy % Uncomment for camera-ready version, but NOT for submission.
\begin{document}


\maketitle


\begin{abstract}
This midterm is a take home - use whatever it takes to solve it. Its not going to be easy!


When you are done, scan the test and send it directly to Brian. typed answers are better.
\end{abstract}

\section{Numerical ODE stability}	

The Forward Euler method for the solution of an ODE of the form
$$ \dot{y} = f(y)$$
reads
$$ y_{j+1} = y_j + hf(y_j). $$
To analyze the stability of the forward Euler method we used the test equation
$$ \dot{y} = \lambda y $$
and wrote the method for the problem as
$$ y_{j+1} = (1 + \lambda h) y_j $$
We then explored for what values of $\lambda h$ the method is stable.

We also explored the backward Euler 
$$ y_{j+1} = y_j + hf(y_{j+1}). $$
and discussed stability for it, which lead to (when applied to the test equation).
$$ y_{j+1} = (1-h\lambda)^{-1} y_j $$

\bigskip

The fractional $\theta$ method reads
$$ y_{j+1} = y_j + h\left(\theta f(y_{j+1}) + (1-\theta) f(y_{j}) \right). $$
For $\theta=0$ we obtain the forward Euler method and for $\theta=1$ we obtain
the backward Euler.
Analyze the stability of the method.

\newpage 
.

\newpage

\section{Match the method to the numerical solution of the ODE}

Assume that you can use three ODE solvers, the forward Euler, the backward Euler and
the midpoint method. Which one should you use for the system of ODE's (and why)
$ \dot{y} = Ay$ where $A$ is 
\begin{enumerate}
\item $ \begin{pmatrix} -2 & 1 \\ 1 & -2 \end{pmatrix} $
\item $ \begin{pmatrix} 0 & 1 \\ -1 & 0 \end{pmatrix} $
\item $ \begin{pmatrix} -10^3 & 1 \\ 1 & -10^-3 \end{pmatrix} $
\end{enumerate}


\newpage

.

\newpage

\section{Fluxes}

The curl of a 3D field is defined by
$$ \nabla \times {\vec J} = \begin{pmatrix}
0 & \partial_z & - \partial_y \\
-\partial_z &  0 & \partial_x \\
\partial_y &  - \partial_x & 0 \\
\end{pmatrix} \begin{pmatrix} j_x \\ j_y \\ j_z \end{pmatrix}. $$

\begin{itemize}
\item Show that $ \nabla \times  (\nabla \rho) = 0$
\item Show that $\nabla \cdot (\nabla \times  \vec J) = 0$
\item Show that $\nabla \times\nabla \times \vec J - \nabla \nabla \cdot \vec J = - \nabla^2 \vec J$
where  $\nabla^2 \vec J = [ \nabla^2 j_x ,  \nabla^2 j_y,  \nabla^2 j_z]$.
\end{itemize} 

 \newpage

. 
\end{document}
